% !TeX program = lualatex
\documentclass[12pt]{ctexart}

\usepackage{geometry}
\geometry{a4paper,margin=1in}

\usepackage{amsmath,amssymb,mathtools,bm}
\usepackage[shortlabels]{enumitem}
\setlist[enumerate]{label=(\alph*)}
\DeclareMathOperator{\Span}{span}

\title{Problem Set 3.1: 3, 10, 15, 17, 18, 30 解答}
\author{Zhiyi Wang}
\date{2025/9/19}

\begin{document}
\maketitle

\section*{3}
\paragraph{(a)} 在 $\mathbb{R}^1$ 中仅保留正数 $x>0$,运算仍为原本的加法和数乘。此集合不是子空间,原因:
\begin{itemize}
  \item 不含零向量:集合中无 $0$,违反公理(3)。
  \item 不含加法逆元:任意 $x>0$ 的逆元 $-x$ 不在集合中,违反公理(4)。
\end{itemize}

\paragraph{(b)} 重定义运算
\[
x\oplus y := xy,\qquad c\odot x := x^{\,c}\quad(x,y>0,\ c\in\mathbb{R}).
\]
检验标量对此加法的分配律(对应原公理(7)):
\[
3\odot(2\oplus 1) \stackrel{?}{=} (3\odot 2)\oplus(3\odot 1).
\]
计算:$2\oplus 1=2\cdot 1=2$,$3\odot 2=2^3=8$,$3\odot 1=1^3=1$,于是左边 $=3\odot 2=8$,右边 $=8\oplus 1=8\cdot 1=8$,相等。此加法下的零向量由乘法单位元给出,即 $e=1$,满足 $x\oplus e=x$。

\section*{10}
\begin{enumerate}[(a)]
  \item 平面 $b_1=b_2$:包含 $\mathbf{0}$,对加法与数乘封闭,是子空间。
  \item 平面 $b_1=1$:不含 $\mathbf{0}$,不是子空间。
  \item 集合 $b_1b_2b_3=0$:不对加法封闭。反例
  \[
  \mathbf{x}=(1,0,1),\ \mathbf{y}=(0,1,1)\ \text{均满足乘积为零,但}\ 
  \mathbf{x}+\mathbf{y}=(1,1,2),\ \ 1\cdot 1\cdot 2\ne 0.
  \]
  不是子空间。
  \item $\Span\{(1,4,0),(2,2,2)\}$:两条线性无关的向量的所有线性组合,是子空间。
  \item 超平面 $b_1+b_2+b_3=0$:线性方程的解空间,是子空间。
  \item 集合 $b_1\le b_2\le b_3$:乘以 $-1$ 会颠倒不等式,不是子空间。
\end{enumerate}

\section*{15}
\paragraph{(a)} 两个过原点的平面在 $\mathbb{R}^3$ 中的交集通常是一条\underline{过原点的直线};当两平面重合时,交集也可能是\underline{该平面本身}。

\paragraph{(b)} 一个过原点的平面与一条过原点的直线的交集通常是\underline{$\{\mathbf{0}\}$};当该直线落在该平面内时,交集也可能是\underline{这条直线本身}。

\paragraph{(c)} $\mathbf{0}\in S$ 且 $\mathbf{0}\in T$,故 $\mathbf{0}\in S\cap T$。若 $\mathbf{x},\mathbf{y}\in S\cap T$,则 $\mathbf{x},\mathbf{y}$ 同时在 $S$ 与 $T$。由于 $S,T$ 各自对加法、数乘封闭,$\mathbf{x}+\mathbf{y}$ 与 $c\mathbf{x}$(任意 $c\in\mathbb{R}$)仍各自在 $S,T$,故在 $S\cap T$。于是 $S\cap T$ 为子空间。

\section*{17}
\paragraph{(a)} 可逆矩阵集合不是子空间。理由:不含零矩阵;且对加法不封闭,例如
\[
I_n\ \text{与}\ -I_n\ \text{均可逆,但}\ I_n+(-I_n)=0\ \text{奇异}.
\]

\paragraph{(b)} 奇异矩阵集合也不是子空间。反例:
\[
A=\begin{pmatrix}1&0\\0&0\end{pmatrix},\quad
B=\begin{pmatrix}0&0\\0&1\end{pmatrix}\ \text{均奇异,但}\ A+B=I_2\ \text{可逆}.\]

\section*{18}
\begin{enumerate}[(a)]
  \item \textbf{真}。若 $A^T=A$ 与 $B^T=B$,则
  \[
  (A+B)^T=A^T+B^T=A+B,\qquad (cA)^T=cA^T=cA,
  \]
  对加法与数乘封闭。
  \item \textbf{真}。若 $A^T=-A$ 与 $B^T=-B$,则
  \[
  (A+B)^T=A^T+B^T=-(A+B),\qquad (cA)^T=cA^T=-cA,
  \]
  亦对加法与数乘封闭。
  \item \textbf{假}。非对称矩阵集合对加法不封闭。举例
  \[
  A=\begin{pmatrix}1&1\\0&0\end{pmatrix},\quad
  B=\begin{pmatrix}1&-1\\0&0\end{pmatrix}\ \text{均非对称,但}\ 
  A+B=\begin{pmatrix}2&0\\0&0\end{pmatrix}\ \text{是对称矩阵}.
  \]
\end{enumerate}

\section*{30}
并无清晰思路,询问ai后有如下答案:
\paragraph{(a)} 定义 $S+T:=\{\,s+t:\ s\in S,\ t\in T\,\}$。有
\[
\mathbf{0}=\mathbf{0}_S+\mathbf{0}_T\in S+T.
\]
若 $x_1=s_1+t_1,\ x_2=s_2+t_2$,则
\[
x_1+x_2=(s_1+s_2)+(t_1+t_2)\in S+T\quad(\text{因 }S,T\text{ 各自对加法封闭}),
\]
任意 $c\in\mathbb{R}$ 有
\[
c\,x_1=c(s_1+t_1)=(cs_1)+(ct_1)\in S+T\quad(\text{因 }S,T\text{ 对数乘封闭}).
\]
故 $S+T$ 为 $V$ 的子空间。

\paragraph{(b)} 若 $S,T$ 是 $\mathbb{R}^m$ 中两条过原点的直线,则
\[
S+T=\begin{cases}
S & \text{若 }S=T,\\[2pt]
\text{过原点的一个平面 }(\Span\{S,T\}) & \text{若 }S\ne T.
\end{cases}
\]
而 $S\cup T$ 只是两条直线的并集,通常对加法不封闭(除非 $S=T$)。由定义可知
\[
\Span(S\cup T)=S+T,
\]
即并集的张成空间正是它们的和空间。
\end{document}
